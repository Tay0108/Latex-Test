% Krzysztof Kosecki %
\documentclass[a4paper,11pt]{article}
\usepackage[a4paper,left=3cm,right=2.5cm,top=2cm,bottom=2cm]{geometry}
\usepackage[utf8]{inputenc}
\usepackage[T1]{fontenc}
\usepackage{amssymb}
\usepackage{amsthm}
\usepackage[english]{babel}
\usepackage{times}
\usepackage{anysize}
\usepackage{titlesec}
\usepackage{fancyhdr}
\usepackage{multirow}
\usepackage{courier}
\usepackage{color}
\usepackage{listings}

\newcommand\tab[1][0.4cm]{\hspace*{#1}}

\setcounter{section}{35}
\setcounter{footnote}{7}
\setcounter{subsection}{1}
\setcounter{table}{14}
\setcounter{page}{151}

\pagestyle{fancy}
\fancyhf{}
\lhead{\ref{r36} \tab LITERATURA}
\renewcommand{\footrulewidth}{0.4pt}
\rfoot{Strona \thepage}

\begin{document}
    \section{Tabela i literatura}
    \label{r36}
        \subsection{Instrukcja}
        Należy napisać document klasy \textit{article}, w postaci takiej jak Pani/Pan trzyma w ręku.
        Dopuszczalne są odstępstwaod tej postaci, tj. rozszerzenia oraz zawężeniado uproszczonej postaci (za dodatkowe "dodatnie" i "ujemne" punkty). Informacje o tych modyfikacjach znajdują się w przypisach\footnote{
        Zalecane parametry dokumentu to \textit{a4paper,11pt}, marginesy: górny i dolny 2cm, lewy 3cm, a prawy 2,5cm.}.\\
        \indent Ta strona zawiera rozdział (sekcję) (\ref{r36}), a w nim dwa podrozdziały (podsekcje)\footnote{
        Najlepiej byłoby nadać numer rozdziału ustawiając odpowiedni licznik. W przypadku innych obiektów --- podobnie, ale należy traktować to jako kosmetykę. Ważniejsze jest, aby w odwołaniach do obiektów używać przypisanych im etykiet. 
        } i spis literatury. \\
		 \indent\color{red}Na początku pliku źródłowego, w komentarzu proszę wpisać swoje imię i nazwisko.
		 Wersję źródłową (.tex i .bib) i wynikową (.pdf) proszę wysłać na adres \texttt{miller@agh.edu.pl}\color{black}\footnote{
		 	Te dwa zdania należy napisać kolorem czerwonym.
		 }.

        \subsection{Zadania}
        \begin{enumerate}
            \item W tabeli (\ref{15}) przedstawiono klasyfikację stałych w języku C.

	\begin{table}[h]
		\centering
		\caption{Stałe w języku C}
		\label{15}
		\begin{tabular}{llll|l|}
			\cline{5-5}
			& & & & Przykład \\ \hline
			\multicolumn{4}{|l|}{Deklarowane stałe} & \ttfamily{const int size =128} \\ \hline
			\multicolumn{4}{|l|}{Stałe preprocesora} & \ttfamily{\#define SIZE 256}   \\ \hline
			\multicolumn{1}{|l|}{\multirow{7}{*}{Literały}} & 
			\multicolumn{3}{l|}{łańcuch znakowy}
			& \ttfamily{"koniec linii.\textbackslash{}n"} \\ \cline{2-5} 
			\multicolumn{1}{|l|}{} & 
			\multicolumn{2}{l|}{\multirow{2}{*}{znakowe}} & \textit{escape sequence} & \ttfamily{'\textbackslash{}n','\textbackslash{}xa4'} \\ \cline{4-5} 
			\multicolumn{1}{|l|}{} & 
			\multicolumn{2}{l|}{} & znak & \ttfamily{'A','!'}  \\ \cline{2-5} 
			\multicolumn{1}{|l|}{}                          & \multicolumn{1}{l|}{\multirow{4}{*}{liczbowe}} & \multicolumn{1}{l|}{\multirow{3}{*}{całkowite}} & dziesiętne & \ttfamily{8743} \\ \cline{4-5} 
			\multicolumn{1}{|l|}{}  &
			\multicolumn{1}{l|}{}  & 
			\multicolumn{1}{l|}{} & ósemkowe & \ttfamily{07464} \\ \cline{4-5} 
			\multicolumn{1}{|l|}{} & 
			\multicolumn{1}{l|}{} & 
			\multicolumn{1}{l|}{}  & szestnastkowe & \ttfamily{0x5AFF} \\ \cline{3-5} 
			\multicolumn{1}{|l|}{} & 
			\multicolumn{1}{l|}{} & 
			\multicolumn{2}{l|}{zmiennoprzecinkowe} & \ttfamily{140.58} \\ 
			\hline
		\end{tabular}
	\end{table}

            \item Należy\footnote{
            Możliwe uproszczenie: znak diamentu w każdej pozycji można zastąpić standardowym znakiem dla "\textit{itemize}".
            }:
              \renewcommand{\labelitemi}{$\diamondsuit$}
              \begin{itemize}
              	\item znaleźć w sieci notki bibliograficzne dwóch pozycji\footnote{
              	Z dokładnością do autorów i tytułu - pozostałe dane mogą wskazywać na inne wydanie.
              	},
              	\item utworzyć plik \ttfamily{.bib}
              	\normalfont % nie wiem czemu tutaj to jest wymagane
              	\item odwołać się do znalezinych pozycji cytując zdanie: \\
              	\indent Dalsze informacje można znaleźć w literaturze \cite{bulirsch:02} lub \cite{Kernighan:1988:CPL:576122}.
              \end{itemize}
        \end{enumerate}
        
        \renewcommand\refname{Literatura}
        \bibliographystyle{plain}
        \bibliography{przypisy}

      
\end{document}
